\documentclass[a4paper, titlepage, 10pt]{article}
\usepackage[T2A]{fontenc}
\usepackage[utf8]{inputenc}
\usepackage[english, russian]{babel}
\usepackage{indentfirst}
\usepackage{tabularx}
\usepackage{amsmath}
\usepackage{graphicx}
\usepackage{float}
\hoffset=-2.1cm
\voffset=-1.6cm
\setlength{\parindent}{1cm}
\textwidth= 16cm
\textheight = 24cm
\newcommand{\HRule}{\rule{\linewidth}{0.5mm}}
\newcommand{\textunderscript}[1]{$_{\text{#1}}$}
\begin{document}
\tableofcontents
\newpage
\section{Техническое задание.}
\indent Реализовать программу, производящую имитационое моделирование передачи сигнлов в ридиоканале с множеством приёмников и передатчиков. Вторичной задачей является уменьшение времени работы программы.
\appendix
\section{Описание модели канала.}
При написании программы использовалась модель со следующими праметрами:
\begin{enumerate}
  \item{Зависимость мощности сигнала от расстоянии расчитывается по следующему уравнению:\\
  \begin{center}
    $
    L = \frac{100*P}{d^5 + a}
    $
  \end{center}
  где P - Мощность передатчика.\\
  d - Расстояние между передатчиком и приёмником.\\
  a - Зачение силы затухания в точке нахождения приёмника.\\
  }
  \item{Значения переменной а в разных точках определяется матрицей случайных гауссовских величин, корреллированных в пространстве.}
\end{enumerate}
\begin{figure}[h!]
 \begin{center}
  \includegraphics[width=0.7\textwidth]{slow_fading_1.jpg}
  \caption{Отображение примера матрицы значений a в тонах серого.}
 \end{center}
\end{figure}
\addcontentsline{toc}{figure}{Отображение примера матрицы значений a в тонах серого.}

\section{Описание работы программы.}
\indent На вход программе подаются параметры моделирования, такие как:
\begin{enumerate}
 \item{Колчество приёмников и передатчиков.}
 \item{Размер решётки для моделирования, высота/ширина.}
 \item{Использование графического вывода.}
 \item{Использование вывода в файл с указанием названия файла.}
 \item{Ввод из файла приёмников и передатчиков.}
 \item{Запрос краткой справки.}
\end{enumerate}
\indent Генерирование координат источников и передатчиков происходит по равномерному закону распределения. На начальной стадии разработки каждому передатчику сопоставлялся приёмник (абонент). Так, при расчёте проверялось, принадлежит ли абонент данному передатчику. Если принадлежит, то мощность передатчика относилась к сигналу, если абонент не принадлежал передатчику, то мощность относилась к шуму. Результатом моделирования являются уровни отношения сигнал/шум для каждого из абонентов. При генерирования матрицы величин "a" использовалась нереалистичная модель, т.е. некоррелированные гауссовские величины.
\indent При попытке ускорения работы программы был использован метод параллельных вычислений. Была использована библиотека Posix Threads и OpenMP. При написании первой версии поточной программы не было необходимости в использовании механизмов защиты памяти от "гонок" потоков, так как запись происходила в разные сегменты памяти. Однако, ускорения работы программы не толко не произошло, программа начала работать медленнее. Была рассмотрена возможная причина такого поведения программы, заключавшаяся в конфликте при считывании данных из памяти. Были так же рассмотрены несколько решений данной проблемы:
\begin{enumerate} 
 \item{Выделять дополнительную память для копий конфликтных областей в отдельных потоках, передавая указатели на области памяти в потоки. Данный способ позволило выделить память с разными адресами, однако к желаемому результату это не привело.}
 \item{Выделять память страницами, т.е., получить размер страницы памяти, выделить необходимое кол-во страниц для хранения массива и скопировать массив в данную область. Указатель на выделенную область далее передаётся в поток, к ускорению программы это также не привело.}
\end{enumerate}
\indent В процессе решения данной проблемы я пытался рассмотреть такой аспект как человеческий фактор, т.е. ошибка при программировании распределения потоков. Я воспользовался библиотекой OpenMP, так как задание параметров распараллеливания в ней производится весьма простым методом (при помощи директивы компилятора \#pragma). Но, к сожалению, данная попытка обернулась неудачей.
\indent В программе реализован вывод на экран приемников и передатчиков. Красная антенна соответствует передатчику, жёлтая приёмнику, серая линия, соединяющяя их, показывает, какой приёмник соответствует передатчику и наоборот.
\section{Выводы по проделанной работе.}
\indent В данный момент возможны два направления развити решения задачи. Первое - поиск ошибки в коде и её исправление (если она, конечно, там есть), второе напрявление - написание функции вычисления мощности сигнала для OpenCL. Так же можно реализовать функцию подключения множества абонентов к одному передатчику.
\end{document}
