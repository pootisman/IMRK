\documentclass[a4paper, titlepage, 10pt]{article}
\usepackage[T2A]{fontenc}
\usepackage[utf8]{inputenc}
\usepackage[english, russian]{babel}
\usepackage{indentfirst}
\usepackage{tabularx}
\usepackage{amsmath}
\usepackage{graphicx}
\usepackage{float}
\hoffset=-2.1cm
\voffset=-1.6cm
\setlength{\parindent}{1cm}
\textwidth= 16cm
\textheight = 24cm
\newcommand{\HRule}{\rule{\linewidth}{0.5mm}}
\newcommand{\textunderscript}[1]{$_{\text{#1}}$}
\begin{document}
\tableofcontents
\newpage
\section{Техническое задание.}
\indent Реализовать программу, производящую имитационое моделирование передачи сигнлов в ридиоканале с множеством приёмников и передатчиков. Вторичной задачей является уменьшение времени работы программы.
\appendix
\section{Описание модели канала.}
При написании программы использовалась модель со следующими праметрами:
\begin{enumerate}
  \item{Зависимость мощности сигнала от расстоянии расчитывается по следующему уравнению:\\
  \begin{center}
    $
    L = \frac{100*P}{d^5 + a}
    $
  \end{center}
  где P - Мощность передатчика.\\
  d - Расстояние между передатчиком и приёмником.\\
  a - Зачение силы затухания в точке нахождения приёмника.\\
  }
  \item{Значения переменной а в разных точках определяется матрицей случайных гауссовских величин, корреллированных в пространстве.}
\end{enumerate}
\begin{figure}[h!]
 \begin{center}
  \includegraphics[width=0.7\textwidth]{slow_fading_1.jpg}
  \caption{Отображение примера матрицы значений a в тонах серого.}
 \end{center}
\end{figure}
\addcontentsline{toc}{figure}{Отображение примера матрицы значений a в тонах серого.}

\section{Описание работы программы.}
\indent На вход программе подаются параметры моделирования, такие как:
\begin{enumerate}
 \item{Колчество приёмников и передатчиков.}
 \item{Размер решётки для моделирования, высота/ширина.}
 \item{Использование графического вывода.}
 \item{Использование вывода в файл с указанием названия файла.}
 \item{Ввод из файла приёмников и передатчиков.}
 \item{Запрос краткой справки.}
\end{enumerate}
\indent Генерирование координат источников и передатчиков происходит по равномерному закону распределения. 
\section*{Выводы по проделанной работе.}
\end{document}
